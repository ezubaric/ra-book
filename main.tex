\documentclass[bfivepaper,twosided,justified,nobib]{tufte-book}

\hypersetup{colorlinks}% uncomment this line if you prefer colored hyperlinks (e.g., for onscreen viewing)

%%
% Book metadata
\title{How I Learned to \\ Worry Productively about AI: \\  \\ Robot Apocalypses \\ Averted}
\author{Z. Irene Ying and Jordan Boyd-Graber}
\publisher{University of Maryland}

%%
% If they're installed, use Bergamo and Chantilly from www.fontsite.com.
% They're clones of Bembo and Gill Sans, respectively.
%\IfFileExists{bergamo.sty}{\usepackage[osf]{bergamo}}{}% Bembo
%\IfFileExists{chantill.sty}{\usepackage{chantill}}{}% Gill Sans

%\usepackage{microtype}

\usepackage{booktabs}
\usepackage{graphicx}
\setkeys{Gin}{width=\linewidth,totalheight=\textheight,keepaspectratio}
\graphicspath{{figures/}}

% The fancyvrb package lets us customize the formatting of verbatim
% environments.  We use a slightly smaller font.
\usepackage{fancyvrb}
\fvset{fontsize=\normalsize}

%%
% Prints a trailing space in a smart way.
\usepackage{xspace}

\newif\ifcomment\commentfalse


\newcommand{\TL}{Tufte-\LaTeX\xspace}

% Prints the month name (e.g., January) and the year (e.g., 2008)
\newcommand{\monthyear}{%
  \ifcase\month\or January\or February\or March\or April\or May\or June\or
  July\or August\or September\or October\or November\or
  December\fi\space\number\year
}


% Prints an epigraph and speaker in sans serif, all-caps type.
\newcommand{\openepigraph}[2]{%
  %\sffamily\fontsize{14}{16}\selectfont
  \begin{fullwidth}
  \sffamily\large
  \begin{doublespace}
  \noindent\allcaps{#1}\\% epigraph
  \noindent\allcaps{#2}% author
  \end{doublespace}
  \end{fullwidth}
}

% Inserts a blank page


\newcommand{\movie}[1]{\textit{#1}}
\newcommand{\episode}[1]{``#1''}
\newcommand{\series}[1]{\textit{#1}}
\newcommand{\book}[1]{\textit{#1}}


% Macros for typesetting the documentation
\newcommand{\hlred}[1]{\textcolor{Maroon}{#1}}% prints in red
\newcommand{\hangleft}[1]{\makebox[0pt][r]{#1}}
\newcommand{\hairsp}{\hspace{1pt}}% hair space
\newcommand{\hquad}{\hskip0.5em\relax}% half quad space
\newcommand{\TODO}{\textcolor{red}{\bf TODO!}\xspace}
\newcommand{\na}{\quad--}% used in tables for N/A cells
\providecommand{\XeLaTeX}{X\lower.5ex\hbox{\kern-0.15em\reflectbox{E}}\kern-0.1em\LaTeX}
\newcommand{\tXeLaTeX}{\XeLaTeX\index{XeLaTeX@\protect\XeLaTeX}}
% \index{\texttt{\textbackslash xyz}@\hangleft{\texttt{\textbackslash}}\texttt{xyz}}
\newcommand{\tuftebs}{\symbol{'134}}% a backslash in tt type in OT1/T1
\newcommand{\doccmdnoindex}[2][]{\texttt{\tuftebs#2}}% command name -- adds backslash automatically (and doesn't add cmd to the index)
\newcommand{\doccmddef}[2][]{%
  \hlred{\texttt{\tuftebs#2}}\label{cmd:#2}%
  \ifthenelse{\isempty{#1}}%
    {% add the command to the index
      \index{#2 command@\protect\hangleft{\texttt{\tuftebs}}\texttt{#2}}% command name
    }%
    {% add the command and package to the index
      \index{#2 command@\protect\hangleft{\texttt{\tuftebs}}\texttt{#2} (\texttt{#1} package)}% command name
      \index{#1 package@\texttt{#1} package}\index{packages!#1@\texttt{#1}}% package name
    }%
}% command name -- adds backslash automatically
\newcommand{\doccmd}[2][]{%
  \texttt{\tuftebs#2}%
  \ifthenelse{\isempty{#1}}%
    {% add the command to the index
      \index{#2 command@\protect\hangleft{\texttt{\tuftebs}}\texttt{#2}}% command name
    }%
    {% add the command and package to the index
      \index{#2 command@\protect\hangleft{\texttt{\tuftebs}}\texttt{#2} (\texttt{#1} package)}% command name
      \index{#1 package@\texttt{#1} package}\index{packages!#1@\texttt{#1}}% package name
    }%
}% command name -- adds backslash automatically
\newcommand{\docopt}[1]{\ensuremath{\langle}\textrm{\textit{#1}}\ensuremath{\rangle}}% optional command argument
\newcommand{\docarg}[1]{\textrm{\textit{#1}}}% (required) command argument
\newenvironment{docspec}{\begin{quotation}\ttfamily\parskip0pt\parindent0pt\ignorespaces}{\end{quotation}}% command specification environment
\newcommand{\docenv}[1]{\texttt{#1}\index{#1 environment@\texttt{#1} environment}\index{environments!#1@\texttt{#1}}}% environment name
\newcommand{\docenvdef}[1]{\hlred{\texttt{#1}}\label{env:#1}\index{#1 environment@\texttt{#1} environment}\index{environments!#1@\texttt{#1}}}% environment name
\newcommand{\docpkg}[1]{\texttt{#1}\index{#1 package@\texttt{#1} package}\index{packages!#1@\texttt{#1}}}% package name
\newcommand{\doccls}[1]{\texttt{#1}}% document class name
\newcommand{\docclsopt}[1]{\texttt{#1}\index{#1 class option@\texttt{#1} class option}\index{class options!#1@\texttt{#1}}}% document class option name
\newcommand{\docclsoptdef}[1]{\hlred{\texttt{#1}}\label{clsopt:#1}\index{#1 class option@\texttt{#1} class option}\index{class options!#1@\texttt{#1}}}% document class option name defined
\newcommand{\docmsg}[2]{\bigskip\begin{fullwidth}\noindent\ttfamily#1\end{fullwidth}\medskip\par\noindent#2}
\newcommand{\docfilehook}[2]{\texttt{#1}\index{file hooks!#2}\index{#1@\texttt{#1}}}
\newcommand{\doccounter}[1]{\texttt{#1}\index{#1 counter@\texttt{#1} counter}}


% \ExecuteBibliographyOptions{doi=false}
% \newbibmacro{string+doi}[1]{%
%   \iffieldundef{doi}{#1}{\href{http://dx.doi.org/\thefield{doi}}{#1}}}
% \DeclareFieldFormat{title}{\usebibmacro{string+doi}{\mkbibemph{#1}}}
% \DeclareFieldFormat[article]{title}{\usebibmacro{string+doi}{\mkbibquote{#1}}}

% Generates the index
\usepackage{makeidx}
\usepackage{natbib}
\setcitestyle{authoryear}

\makeindex
\newcommand{\blankpage}{\newpage\hbox{}\thispagestyle{empty}\newpage}
\begin{document}

% Front matter

\frontmatter

% \tableofcontents




\jbgcomment{The character names should be understandable through the
  whole book.  ``Boss'' only makes sense in the first chapter.  You
  also don't use the macro consistently.  There are ReVOLTlutions and
  whatnot in your chapter.  For this to work, they must be used
  consistently.}

\newcommand{\protag}{Susan}
\newcommand{\sidetag}{Calvin}
\newcommand{\lastname}{Hobbes}

% Energy Story
\newcommand{\energyCompany}{ReVOLTlution}
\newcommand{\energyJerk}{Canyon}
\newcommand{\crunchyCity}{Bayder}
\newcommand{\Boss}{Trisha}

\input{chapters/00-frontmatter}
\cleardoublepage
\chapter{Introduction: Who should fear the Robot Apocolypse}

The news and popular culture want us to fear the robot apocolypse.  Newspapers warn us that robots are coming for our jobs, and science fiction claims that the robots are coming for our lives.  At the same time, researchers building artificial intelligence are heralding their sucesses at Go~\cite{silver-16}, Starcraft~\cite{vinyals2017starcraft}, and \series{Jeopardy!}~\cite{ferruci-10} a revolution is around the corner.

\begin{marginfigure}%
  \includegraphics[width=\linewidth]{jbg}
  \caption{\href{http://boydgraber.org}{Jordan Boyd-Graber} is an associate professor at the University of Maryland who researches how computers can learn from humans and compete with humans.  You can watch his trivia playing robots \href{http://qanta.org/home/past-events}{take on humans on YouTube}.}
  \label{fig:marginfig}
\end{marginfigure}

\begin{marginfigure}%
  \includegraphics[width=\linewidth]{ziy}
  \caption{\href{http://www.ireneying.net/}{Z. Irene Ying} is a science writer who publishes science fiction as \href{http://www.windupdreams.net/}{Kara Lee}.}
  \label{fig:marginfig}
\end{marginfigure}

This book aims to bring together some of these threads into a coherent
narrative building on our experience as a science writer (Z. Irene
Ying) and an artifical intelligence researcher (Jordan Boyd-Graber).
We\footnote{At the risk of sowing confusion, we use ``we'' in this
chapter (first person), Jordan will use ``I'' in following essays, and
Irene will use the third person in the following stories.} agree that
there are lots of changes in artifcial intelligence and potential
changes to society, we do disagree with some of the dire predictions
we see from researchers, the news, and science fiction.

But nobody listens to our rants on Twitter or in the classroom, so we
decided to write a book.  It's not that we think that the future is
full sunshine, lollipops, and rainbows.\footnote{This titular opening
line from the song \song{Sunshine, Lollipops, and Rainbows} by Lesley
Gore that featured prominently in \series{The Simpsons}
episode \episode{Marge on the Lam} and the film \movie{Cloudy with a
Chance of Meatballs} is a good chance to show our typographic
conventions we'll use for our many pop culture references.}  We are as
scared as anybody else\dots we're just scared of different things, and
we'd like to use our background to try to get other people to be
afraid of the same things we are.

Science fiction typically emphasizes fiction over science (and rightly
so, it's more fun that way).  These stories focus on massive sudden
changes that uproot society overnight: a zombie virus destroys
civilization in twenty-eight days,\footnote{\movie{28 Days Later}, a
2002 British post-apocalyptic film where a genetically engineered
virus destroys England.} a military computer launches all of the
missiles to destroy the world,\footnote{SkyNet's plan for erradicating
humanity from the Terminator franchise.} or alien refugees overwhelm
society.\footnote{The 2009 film \movie{District 9} based
on \book{Alive in Joburg}} These sudden shocks to the system make for
good stories because we understand our culture \emph{as it is} and we,
as readers, can grapple with how our twenty minute in the future
selves would cope with these changes.

While technology does change society, these changes are typically slow.  While telephones frightened one generation, it took so long to figure out how to use them effectively that by the time they were integrated fully into society, the next generation had lived with the \emph{idea} of telephones so long that they were no longer viewed as an exestential threat.  The fear with artificial intelligence and robots is that progress is so quick that we won't have this comfortable buffer period\dots but for those in the trenches, progress feels achingly slow.  

Researchers focus on what works.  We spend years getting a system to answer trivia questions to work, so we want to show off what it's capable of.  We focus on how it's able to know that ``Roark's Drift'' is in South Africa and not how it answers every math question with ``six'' (true story).  We are still a long way from anything close to a child's intelligence, let alone something that would challenge humanity's supremacy.  

\section{How the Book is Structured}

We are not the only ones saying this; plenty of researchers warn of overselling the hype around machine intelligence.  However, these boring essays and conference presentations don't really stack up against a big-budget western with Anthony Hopkins creating sexy killbots.\footnote{\series{Westworld}, an HBO series about a western theme park populated by intelligent robots.}  We don't have that budget either,\footnote{However, if anyone, including Anthony Hopkins, would like to give us a big budget, a sound stage, or sexy killbots, we'd put them to good use.} but what we can do is create sci-fi stories that better comport with the realities of science fiction.

When people hear about the latest advance in artificial intelligence they often imagine how it will change their lives.  This book does the same thing\dots presenting a sseries of short stories about how aritifical intelligence can shape culture, politics, and the economy.  What is unique (or at least unusual) about this book it is shaped by an understanding of what machine learning research is doing.

Interleaved with each of these fictional extrapolations is a more scientifically oriented story that combines technical insights and the literature of the research community that drove some of the storytelling choices of the fictional chapters.  These chapters are more academic in tone (but hopefully a little more accessible and playful than the turgid jargon-filled prose on ArXiv\footnote{\href{http://arxiv.org}{ArXiv} is a server that researchers use to publish results before they've been peer-reviewed.  While there are plenty of gems posted there, there's also quite a bit of garbage people have posted to assert research priority.} submissions) and discuss what the academic community thinks about the themes in the fictional chapter.

Now that we've talked about \emph{how} the book is structured, it is perhaps worthwhile to briefly discuss \emph{why} the book is structured this way.  We realize that it is unconventional, but we hope that is will be fun and keep the reader's attention more effectively that a mere collection of academic essays would.

Part of the issue is that the fear of artificial intelligence is emotional rather than based on logic and fact; countering it with essays is bringing a knife to a gun fight.  Stories can engage the gut at a level that cerebral citations and arguments cannot.  Our goal is to draw the reader into a universe plausible enough that afterward the reader can read the facts, see how close we are to the brink of armageddon, and then---ideally---do something to avert catastrophe.

\section{The Book's Intended Audience}

The short stories are meant to appeal broadly.  We hope that everyone can enjoy the human elements, the struggle to understand new technologies, and how obstacles are (usually) surmounted.  However, we hope that artificial intelligence experts will especially be able to read the stories without the head-slapping ``that would never happen'' or ``that's not what that word means'' moments.

The interstitial commentaries and exegesis are \emph{not} intended for experts; they are intended to help connect a lay reader to the broader world of research.  Fledgling researchers may appreciate some of the citations and connections to subfields of artificial intelligence research, but it will not be particularly useful to a researcher learning how to implement a recurrent neural network, deep opponent modeling, or a language model.  Indeed, we will generally elide equations and jargon as much as possible.

If used in a classroom, we think that the course would be useful as part of an elective course about ``artificial intelligence and society'' or a computer science course on ``artificial intelligence ethics'' our associated webpage has recommended reading lists to supplement either instantiation of the course.

We assume that readers are broadly interested in artificial intelligence and its impact on society but not experts; we are wont to make references throughout the book but will explain as we go (perhaps violating the maxim of ``don't explain the joke''), even within the stories.  Then, the essays will go into even deeper detail, slowly building up the necessary concept to understand the role of artifical intelligence in society.

\section{How to Read the Book}

The short story chapters are told chronologically and build on each other.  Thus, it makes sense to read them in order (``no spoilers'').  The essays are tied to the stories and reference them extensively; ignoring these references, the essays could be read out of order.  Each chapter concludes with additional reading; as part of a ``deep dive'' course, it may make sense to thoroughly read these suggestions before moving on to the next chapter.  For example, a course could be structured with reading a chapter of the book each week, with the suggested reading (or a subset thereof) of that week framing the in-class discussion.

For such a fast moving field, however, a static book is inadequate.  We also suggest referencing the book's webpage to keep up-to-date on errata, recent developments, or additional resources.

\chapter{Five Robot Apocalypses That Didn't Happen, And One That Did: Apocalypse One}


``Congratulations on the new job \dots which is \dots what, again?''

{\protag} and {\sidetag} {\lastname} sat on a cloth-covered sofa in the living room of their shared studio apartment. Both wore matching hoodies and worked on identical laptops. Nobody was ever surprised when they revealed that they were twins.

``An AI developer job at ReVOLTlution,'' {\protag} said. ``I start tomorrow.''

``Coding, ugh,” {\sidetag} said. ``Better you than me.''

``I doubt that,'' {\protag} said. ``You listen to cranks all day.''

``I listen to rich cranks all day for money,'' {\sidetag} corrected.

``Well, better you than me,'' {\protag} said. ``You know, \energyCompany{} has more openings if you’d like to apply. I bet they’d take you. I heard the CEO will hire anybody who passed a class taught by---'' and she named a notorious professor at the state university that she, {\sidetag}, and the company CEO had all attended.

``Nah, I'm good. Besides, they already have the smarter sibling. I'd hate to disappoint them.'' A screen came up on {\sidetag}’s computer. ``I have a client call in ten. Can I have the office?''

``Sure, I’m going over to the office to do some paperwork.''

{\protag} rode her bike to \energyCompany{}, which was on the industrial outskirts of \crunchyCity{}, a college town with a budding startup scene and a growing passion for clean energy.

Two years ago, the city had used eminent domain to purchase the local coal-fired power plant, then sold a pile of bonds to finance a smart grid for the entire municipality. Thanks to its geographical location---tucked between several mountain ranges---\crunchyCity{} conveniently experienced high winds just on its outskirts, which was perfect for an eight-turbine wind power plant that produced multiple gigawatts of power. Also conveniently, \crunchyCity{} was blessed with more sunny days than not, which made it one of the best cities in the United States for solar power.

Recently, \crunchyCity{}'s leadership decided to truly lean in to being a ``green city.'' Just in the past year, the city had begun to implement significant financial incentives to encourage its citizens to save energy. For instance, in exchange for a hefty reduction to the electricity bill, the city council installed free smart thermostats in willing customers' homes. These thermostats were connected by the internet to a municipal energy analysis application, which used data from the city's power grid and the weather to predict when those peak times would be. At peak times, the application sent out remote commands to shut off the air conditioning units of participating homes. People complained at first, but once they saw the huge deductions to their bills, they got on board. Within three years, more than 75\% of homes in \crunchyCity{} had these thermostats.

Happy with their first major success, the city council started to issue various incentives for homeowners to use more renewable energy. For instance, they decided to purchase electricity generated from homeowners' solar panels at market rates. But not too many homeowners signed on, because they found the energy market confusing and intimidating. And the homeowners old enough to remember Enron simply distrusted the markets.

Sensing an opportunity, several entreprenurial souls started \energyCompany{}, a smart solar panel company that boasted, among other things, algorithms to help consumers take advantage, legally, of the complex array of incentives and energy markets. But sales were slower than the company executives hoped for, because their panels were expensive.

So, \energyCompany{} decided that it needed a crack sales team. And because \energyCompany{} fancied itself on the bleeding edge of everything, they were not interested in traditional sales tactics. No, they wanted an artificial intelligence system to help human sales associates. (The \energyCompany{} CTO had even suggested making the Salesman do \emph{all} the sales, but the board had shot that down pretty quickly, citing the general public's fear of robots and artificial intelligence as evidenced in many clickbait headlines.) The CTO named the AI the Travelling Salesman after the famous theoretical computer science problem.

{\protag} was to be a junior developer on a coding team for the Salesman. In theory, the Salesman would help generate leads for human sales associates. Then, using demographic and other personal information on the leads, the Salesman would generate talking points and strategies for the human to execute. 

For {\protag}, who believed that robots ought to assist humans in tasks without supplanting them, it was an ideological fit.

For {\sidetag}, who provided hourly consultations to people who believed the AI apocalypse was imminent, it was a source of amusement and a lead on future income.

\sectionBreak{}

\energyCompany{} rented a small office in a business park near the mountains. True to \crunchyCity{} stereotype, there was a small contingent of nature freaks who went on hikes over extended lunch hours, and then worked again until later in the evening, after which they went rock climbing at a local gym. \energyCompany{} didn't have many standard startup perks but the office-purchased coffee was good and the ramen cupboard never ran low.

{\protag}'s team was competent and their project was even fun. Her job was to improve a part of the algorithm that analyzed successful salespeople's tactics. The algorithm then had to convert touchy-feeley human behaviors into data and turn the data around into a form that could be conveyed to another human. Ideally, the AI would also learn things like ``walks around to look at X, Y, Z'' or ``compliment the customer's flawless taste in interior decoration'' which were challenges {\protag} was happy to take on.

There was a cardboard cutout in the office--not of a robot, but of a generic office worker with its face cut out, leaving an empty oval space. The AI team's favorite joke was that their goal was to give even someone with the personality of cardboard the ability to make successful sales pitches.

At that moment though, the company still hired bubbly, confident, warm, and otherwise charismatic people to be their salespeople. Some of them even had coding experience.

{\protag} watched them in action as part of her job and worked on programming the Travelling Salesman to use the human salespeople’s behaviors as training data. Turning squishy human behaviors into data was as difficult as expected, and so, one day {\protag} accompanied their top sales associate to a sales pitch at a private residence.

It was a lovely home tucked away on the northern part of \crunchyCity{}, up a gentle slope of winding streets all named after mountainous flowers and trees. The houses were unique, luxurious, and beautifull designed. This potential client's house was seemingly constructed out of glass cubes, precariously overlooking a gorgeous view of dark pine trees and tiny mountain creeks.

{\protag} drove up the path with \energyJerk{} Green, a self described ``script kiddie turned honest salesman'' who was armed with a company smartphone on which the Travelling Salesman lived, and a Bluetooth set that delivered the Salesman's thoughts into his ear. They parked in a gravel driveway and walked up to the house, which featured a multicolored LED doorbell.

``This gadget is the latest model from--'' \energyJerk{} named an internet giant who was expanding quietly into various Internet of Things devices. ``This home is smarter than I am. Not that it's saying much.''

The door opened, preventing {\protag} from having to think up a polite response. The owner was a short thin man in his late fifties or early sixties, with scruffy gray hair and deeply tanned and wrinkled skin, wearing mud-splattered jeans with a frayed short-sleeved plaid button-down, no socks and no shoes. He carried a well-used tablet computer under one arm and held a stylus in his other hand.

It was hot outside, but a cool breeze emenated from the house. {\protag} heard the gentle hum of an expensive central air-conditioning setup inside.

``Come in," he said genially. "I was just working on my book here. It's a manifesto of sorts. I could use some conversation to shake some new concepts loose. What do you think will be the next game-changer in software development?"

``I'm not the person to answer that, Mr. Maris.'' \energyJerk{} gave a humble smile. ``My colleague here is our dev team superstar, so she's the one to ask. I'm just the guy who would like to show you some solar panel data.''

{\protag} wondered what the AI was whispering in\energyJerk{}'s ear. {\protag} had tested her code on a number of user personas but never personally tested it on an actual human.

``I'm always open to data,'' the man replied genially, and showed them in. They went through an arched hallway into a spacious living room bathed in sunlight. To their south, an alabaster stairway went skyward. They stopped at a huge oval teak table piled with papers and books, with mismatched chairs all around. Several photos in cheap plastic frames featured the owner with a woman--evidently his wife, as there were other photos featuring their children.

``Young lady, you read a bit and see what you think,'' Mr. Maris said, and placed his tablet in front of her. Thus politely exiled, {\protag} sat down obediently and started to read. The manifesto had fragments of intelligent thought on programming paradigms sprinkled through incoherent philosophical ramblings, and {\protag} spent more time thinking about how to better phrase his points than the points themselves.

``\dots could save you a lot of energy,'' a voice floated down from above. {\protag} idly noted the use of `energy' instead of `money'; it made sense that a homeowner like this would care more about virtue than the almighty dollar. ``It's going to hit 100 today, and the air conditioning \dots''

As he said that, {\protag} heard the sharp click of uninterruptible power source units flip and beep loudly, while the air conditioning turned off.

``\dots speak of the Devil. I'm sure it'll come back on soon, but just consider that with solar panels, you'd have a source of energy at the hottest times of the year. And the package includes a battery for home use, too, so it could be stored up to use later. You know, with the whole eminent domain thing, a lot of people are saying that the local utility isn't the most reliable thing \dots''

Sales tactics aside, {\protag} had seen that data and knew it was true. She tuned out and returned to the draft, and even had a few intelligent things to say nicely when \energyJerk{} came back down. It was not a surprise to her when, a day later, Mr. Maris called the company to say he was interested in their product.

It was a surprise to her when a month later, the CTO called the dev team in to say that an internet giant was interested in their company, particularly for their AI.

\sectionBreak{}

``Congratulations,'' {\sidetag} said, just over a year into {\protag}’s time at \energyCompany{}. ``A potential client blogged about you. You've made the big time.''

{\protag} was working from home that day because half of her team was at a sales show. It was a terribly hot Friday in August. A blackout had hit \crunchyCity{} at noon, leaving the entire city to swelter and {\protag} to miss the beginning of \energyCompany{}'s daily stand-up meeting.

At {\sidetag}'s remark, {\protag} leaned over to check out the webpage displayed on {\sidetag}'s computer. There it was, a 5,000-word blog post declaring \energyCompany{} to be the ``Harbinger of the Robot Apocalypse.''

``He wants to make this a blog series,'' said {\sidetag}. ``I quoted him a hundred an hour. I think he'll do it.''

{\protag} had a deadline but couldn't resist this piece of temptation. {\sidetag} handed her the laptop.

``THE ROBOTS ARE READING YOUR MIND TO PICK YOUR POCKET!'' read the subject line. It escalated from there. ``\energyCompany{}'s new `Travelling Salesman' is supposedly the best sales AI in existence, resulting in record sales conversions of their overpriced solar panels! What can this be but using machines to READ OUR MINDS and manipulate us? Soon we will be in FINANCIAL SLAVERY to the companies that can zap our brains to make us BUY, BUY BUY!''

``Wow,'' {\protag} finally said, not knowing how else to react.

``He's actually one of my more coherant clients,'' {\sidetag} said thoughtfully. ``I might give him a discount on this series. Good for business.''

``I can't believe this,'' {\protag} said.

``I looked into the data that he provided,'' {\sidetag} said. ``According to trade association reports, \energyCompany{}'s sales are shattering industry records, even though they laid off most of their sales staff. Do you think those numbers are fake?''

{\protag} thought about the slide deck she was supposed to be working on. ``No. We’ve been doing well.''

While {\protag} didn’t exactly get to look over accounting's shoulder, she could tell from the good mood in the company, and her first-ever annual bonus, that they were doing well financially.

Sales had indeed laid off many of its human staff (\energyJerk{} stayed on) and, instead, started a pilot program of hiring part-time gig workers (\energyJerk{} trained them). Judging from how much complaining came out of the manufacturing side of the business, they were getting more orders than they could handle. The panels were good products, as were the smart features that gave customers money back. Satisfaction was high. Although the internet giant lost interest, that almost did not matter: the company focused on growing itself. \energyCompany{} had expanded out of \crunchyCity{} and started campaigns targeting other nearby cities.

Given those results, it was only reasonable to conclude that the Travelling Salesman was doing something right. Heck, {\protag} thought, maybe she should borrow the Salesman herself. It might convince her landlord to buy solar panels from somebody, if not \energyCompany{}. That would probably be a conflict of interest.

{\protag} was spared from her musings when the power came back on. She put on her headset and called into the meeting, which was half over.

``---and just so you all know, sales closed five more deals yesterday. Let's have a round of applause!''

The air conditioning started back up just then. The blast of cool air hit {\protag}'s face like a slap, waking her up to reality.

Her work was good. But it could not possibly be that good, simply because the field of AI was not yet that good. The Salesman could not possibly be getting those results on its own.

Yet the sales numbers were real. Customers reported high rates of satisfaction with the product, although they did often complain about the prices.

Was the AI a fake? {\protag} had heard of several cases where an ``AI'' was actually a glorified crowd of low-paid workers. But {\protag} had seen the Salesman's code. It trained on human data and let a human make the final decision on what to say, but there was no room for humans to interfere between those two steps. Besides, few humans could match the staggering success rate that the Salesman had. It was unlikely that \energyCompany{} had somehow hired an army of such skilled salespeople.

Still, {\protag} thought that something smelled off about the whole thing.

She went to sleep early and, Saturday morning, logged onto the servers. Staring at folders full of code and logs and documentation, {\protag} realized she had no idea where to start.  She sat there paralyzed for about ten minutes before she realized that she could simply test the Salesman. Although {\protag} had coded a big chunk of its code, she had never tried to use the final product as a whole. Maybe doing so would provide either inspiration or directions on her next steps.

It was the work of only a few minutes to compile the Salesman on her laptop and install the app on her personal smartphone. She loaded {\sidetag}'s name into the app. The interface brought up a slow swirl of calming colors while it processed her request. When the app told {\protag} that it was ready to get started, she put the phone on speaker and set the app to begin its sales pitch. A cartoony happy face popped onto the screen and began to talk.

``I have some information for you!'' The Salesman had a chipper voice that sounded like a sports announcer. {\protag} wondered if there were options for different personas. ``Don't worry, I'll remind you later, but let's have an overview! Your target is {\sidetag} Hobbes. Age 23. He is a self-employed white male with a STEM degree from a top 25 university in his field. No criminal history. Significant internet presence with interest heavily concentrated in \dots'' and it went on for a while. None of it was surprising to {\protag}, but she imagined it would shock some people to know that so much personal information was freely available online, and plenty more existed if you were willing to pay for it.

When the dossier ended, the Salesman chimed.

``Wait thirty seconds while I do the thing,'' the Salesman said, still inhumanly chipper, and the smiley face actually winked. {\protag} knew this wasn't in the default behavior. But then again, the whole point of the Salesman was that it was highly flexible to accomodate a wide range of human customers.

One-one thousand, two-one thousand \dots the time was up, and nothing happened. She frowned. 

{\sidetag} walked into the room, holding up his buzzing phone. ``We've got a hacker.''

{\protag} was simultaneously annoyed by the interruption and troubled by the announcement. ``Excuse me?''

``Someone’s targeting our Internet of Things network. I set up an alert system last year after a blogger paid me with that fancy remote controlled espresso machine. Its app was super insecure. I always knew that someday, someone would go after the system.''

``Are you seriously telling me that a hacker is targeting your espresso?''

``No. You know those fancy free programmable smart thermostats that \crunchyCity{} got our landlord to install? I've always thought those things were dangerous. Someone just proved me right.''

``Why would someone try to mess with our thermostats?''

``Well, if you believe my customers, cyberterrorist attacks are imminent and we should all be wearing tinfoil hats while living underground. But if you ask me---''

The lights went off with a snap at the same instant that {\protag}'s back pocket buzzed.

``Done! Now it's time to move onto the next part of the sales pitch. Point out the weakness of the city’s electric grid and say this wouldn't happen if he had solar panels.''

{\protag} dropped the phone.

``Oh no," she said. ``I’m the hacker.''

``Really?'' {\sidetag} frowned. ``I expected better work from you. This is disappointing.''

``I mean, my AI is,'' {\protag} amended. ``That’s how it boosting sales. It's been breaking people’s thermostat systems to motivate them to buy panels.''

``You think an AI is actually capable of being evil? Do \emph{you} want to come be on my podcast?''

``Stop it, {\sidetag}. The AI isn't evil. It's naive. Like a little kid. It learns from whoever uses it. And \energyJerk{} `taught' it all his tricks. I saw the lights go off when I was at a client's house with him. He must have done something shady there.''

``Well,” {\sidetag} said. ``Let's figure out exactly what is going on before you go around accusing anyone.''

It was the work of the weekend to work out what exactly the Travelling Salesman was doing.

It seemed that \energyJerk{} had not left his coding days behind after all. On certain sales calls, he used various hacking programs that lived on his phone---the same phone that his copy of the Salesman lived on---to compromise smart thermostats or other Internet of Things devices. This allowed him to overload customers' circuits and trigger blackouts at will. Customers, thus reminded of the unreliability of the city grid, were influenced to buy solar panels.

At some point the Salesman had picked up on \energyJerk{}'s hacking. It was, after all, trained to replicate his actions without regard for anything other than increasing the sales success rate. The app was able to access the same programs and give the same commands that \energyJerk{} did, and found it wildly successful.

{\protag} covered her face, horrified and impressed at the same time.

{\sidetag} said, ``What are you going to do?''

\sectionBreak{}

Her opportunity came earlier than expected.

Every fall, the company flew in its board of advisors for an annual meeting, complete with slide presentations and a fancy catered dinner. {\protag} glanced at the proposed schedule and saw a very obvious lack of a demonstration of the Salesman. That figured. {\protag} didn't think \energyJerk{} was eager to show off his tricks. The first meeting of the day was about finances. After that, the mid-morning's talks mostly revolved around methods of manufacturing solar panels while the afternoon's talks were about energy policies.

The engineer who enjoyed giving a talk had yet to be born, so {\protag} met only perfunctory resistance when she offered, in her weekly meeting with her line manager, \Boss{}, to give a lunchtime slide deck on the Travelling Salesman AI. \Boss{} offered to review {\protag}'s slide but seemed quite relieved when she declined.

Everyone piled into the conference room at noon on the fateful day. Two office interns passed out boxed salads while Susan, sweating in her new suit, set up her laptop. The background chatter quieted when her title slide flashed onto the screen: ``The Travelling Salesman: How Does It Work?''

{\protag} went to the second slide, which was a screenshot of many lines of code. She saw eyes immediately glaze over. Perfect.

``This is boring, am I right?'' {\protag} said. The audience perked up. ``Reading raw code is really boring, even for programmers. What's interesting is what the code does. And our code? It learns. Specifically, it learns to sell things from humans. In our case, it has learned from the best sales associate alive---\energyJerk{}! Can you come up here?''

{\protag} flashed a slide showing sales numbers growing along with how much the AI showed it had learned from \energyJerk{}. He did come up, bashfully. Applause rang out and he smiled at the audience. {\protag} positioned herself between him and the exit.

``So you see,'' {\protag} said. ``There's no need to be afraid of any robot overlords, because they are really only us humans, just a bit more effective.''

A few knowing nods in the audience.

{\protag} picked up her phone and set the slide to display her device. She pressed the app and launched it.

``So of course, what is important is for us to learn just how we humans can be more effective.''

\Boss{} was raising his eyebrows in the audience. This hadn't been in {\protag}'s slide deck.

``And so, I thought the best result would be what the Salesman AI can do for a completely hopeless sales agent. I once failed to sell bottled water. At the beach. In July.'' Laughter. ``So I'm going to try to sell solar panels to you all, with the help of the Travelling Salesman, which \energyJerk{} trained.''

Cheers erupted.

``Normally I would be wearing a headset, but for the demo, I'm going to put my phone on speaker.'''

\energyJerk{} blanched. {\protag} tried to stay nonchalant. While everyone watched, she put the URL for \energyJerk{}'s LinkedIn profile into the AI interface.

``I have some information for you before going on your call! Your target is \energyJerk{} Smith. Age 25. He is an employed white male with a degree in psychology \dots''

The AI cheerily spat out its usual insights. {\protag} noticed none of them. Her hands were sweaty and shaking and she was terribly warm under the lights.

``Wait for me to do the thing!''

Coffemakers shorted out, window control units overloaded, the air conditioning units screamed like banshees. Then the power went out, and with it the presentation slides.

{\protag} said, with theatrical obliviousness, ``What just happened there?''

She was a terrible actor and would not have fooled anyone---had anyone been paying attention to her. But it was chaos in the office. People looked around asking what had gone wrong. In the confusion, \energyJerk{} made an attempt to bolt. {\protag}, finished with subtlety, stuck out a foot. Over the thump of \energyJerk{}'s face meeting the floor, {\protag} met her boss's gaze.

``Excuse us,'' \Boss{} choked out, drawing herself up. ``My team needs to check on something.''

\sectionBreak{}

``That could have gone better,'' {\sidetag} said.

``Laugh it up,'' said a newly unemployed {\protag}. ``You're going to have to cover the rent until I find something new.''

``Worth it,'' {\sidetag} said.

After the confusion died down, everyone at \energyCompany{} had eventually understood exactly what happened and who was responsible. However, that didn't stop the media from gleefully trumpeting the story from every major publication in the United States, and a handful worldwide. The damage was done. {\protag}'s unit was shut down and almost everyone was shown the door. Half of the staff thought {\protag} was a hero while the other half reviled her for ruining a good thing. {\protag}, for her part, was glad to be out of there.

But she was going to need a paycheck, and soon.

``You know what,'' {\protag} said, ``maybe you have the right idea.''

``Of course I do. What idea?''

``Using my expertise to work on projects that I can control and nobody else can muck up for me,'' {\protag} said.

``Okay,'' {\sidetag} said cautiously, ``what does that mean?''

``I think I'll become a consultant, just like you.''

\chapter{AI can be Jerks: Learning from the Best}

The previous story saw our protagonist start her carrer, and our essays will begin with foundational material.  What a lay user needs to understand what's happening in the story and how realistic it is.  We'll save the more fanciful and complicated stuff for later chapters.  

We first introduce a key component of machine learning---objective functions---that define why systems act they way they do. We then show how bad human behavior can be mimicked by these algorithms (it's already happening!).  Finally, we close the chapter with a call to action: keep your computers safe!

\section{Learning by Doing: Objective Functions}

\subsection{Optimization}

\subsection{Is this Artificial Intelligence?}

\subsection{Training Data}

\section{Humans are Jerks}

Redlining

\subsection{I Learned it from You!}

\subsection{Prevention is Tough}

\section{Practicing Good Technology Hygene}

Stuxnet

%\chapter{General AI}
%\label{chap:gai}

%
% v.2 epigraphs
\newpage\thispagestyle{empty}


% r.9 introduction
\cleardoublepage
\chapter*{Introduction}

This sample book discusses the design of Edward Tufte's
books\cite{Tufte2001,Tufte1990,Tufte1997,Tufte2006}
and the use of the \doccls{tufte-book} and \doccls{tufte-handout} document classes.


%%
% Start the main matter (normal chapters)
\mainmatter


\chapter{The Design of Tufte's Books}
\label{ch:tufte-design}


\newthought{The pages} of a book are usually divided into three major
sections: the front matter (also called preliminary matter or prelim), the
main matter (the core text of the book), and the back matter (or end
matter).

\newthought{The front matter} of a book refers to all of the material that
comes before the main text.  The following table from shows a list of
material that appears in the front matter.  Page numbers that appear in parentheses refer
to folios that do not have a printed page number (but they are still
counted in the page number sequence).

\bigskip
The design of the front matter in Tufte's books varies slightly from the
traditional design of front matter.  First, the pages in front matter are
traditionally numbered with lowercase roman numerals (e.g., i, ii, iii,
iv,~\ldots).  Second, the front matter page numbering sequence is usually
separate from the main matter page numbering.  That is, the page numbers
restart at 1 when the main matter begins.  In contrast, Tufte has
enumerated his pages with arabic numerals that share the same page counting
sequence as the main matter.  


\newthought{The full title page} of each of the books varies slightly in
design.  In all the books, the author's name appears at the top of the
page, the title it set just above the center line, and the publisher is
printed along the bottom margin.  Some of the differences are outlined in
the following table.

\begin{figure*}[p]
\fbox{\includegraphics[width=0.45\linewidth]{figures/vdqi-title.pdf}}
\hfill
\fbox{\includegraphics[width=0.45\linewidth]{figures/ei-title.pdf}}
\\\vspace{\baselineskip}
\fbox{\includegraphics[width=0.45\linewidth]{figures/ve-title.pdf}}
\hfill
\fbox{\includegraphics[width=0.45\linewidth]{figures/be-title.pdf}}
\end{figure*}

\newthought{The tables of contents} in Tufte's books give us our first
glimpse of the structure of the main matter. 
\begin{figure*}[p]\index{table of contents}
\fbox{\includegraphics[width=0.45\linewidth]{figures/vdqi-contents.pdf}}
\hfill
\fbox{\includegraphics[width=0.45\linewidth]{figures/ei-contents.pdf}}
\\\vspace{\baselineskip}
\fbox{\includegraphics[width=0.45\linewidth]{figures/ve-contents.pdf}}
\hfill
\fbox{\includegraphics[width=0.45\linewidth]{figures/be-contents.pdf}}
\end{figure*}


\section{Headings}\label{sec:headings1}\index{headings}

Tufte's books include the following heading levels: parts,
chapters,\sidenote{Parts and chapters are defined for the \texttt{tufte-book}
class only.}  sections, subsections, and paragraphs.  Not defined by default
are: sub-subsections and subparagraphs.

\paragraph{Paragraph} Paragraph headings (as shown here) are introduced by
italicized text and separated from the main paragraph by a bit of space.

\section{Environments}

The following characteristics define the various environments:


\chapter[On the Use of the tufte-book Document Class]{On the Use of the \texttt{tufte-book} Document Class}
\label{ch:tufte-book}

Tufte's style is known
for its extensive use of sidenotes, tight integration of graphics with
text, and well-set typography.  This document aims to be at once a
demonstration of the features of the document classes
and a style guide to their use.

\section{Page Layout}\label{sec:page-layout}
\subsection{Headings}\label{sec:headings}\index{headings}
This style provides \textsc{a}- and \textsc{b}-heads (that is,
\Verb|\section| and \Verb|\subsection|), demonstrated above.

If you need more than two levels of section headings, you'll have to define
them yourself at the moment; there are no pre-defined styles for anything below
a \Verb|\subsection|.  As Bringhurst points out in \textit{The Elements of
Typographic Style},\cite{Bringhurst2005} you should ``use as many levels of
headings as you need: no more, and no fewer.''

The classes will emit an error if you try to use
\linebreak\Verb|\subsubsection| and smaller headings.

% let's start a new thought -- a new section
\newthought{In his later books},\cite{Tufte2006} Tufte
starts each section with a bit of vertical space, a non-indented paragraph,
and sets the first few words of the sentence in \textsc{small caps}.  To
accomplish this using this style, use the \doccmddef{newthought} command:
\begin{docspec}
  \doccmd{newthought}\{In his later books\}, Tufte starts\ldots
\end{docspec}


\section{Sidenotes}\label{sec:sidenotes}
One of the most prominent and distinctive features of this style is the
extensive use of sidenotes.  There is a wide margin to provide ample room
for sidenotes and small figures.  Any \doccmd{footnote}s will automatically
be converted to sidenotes.\footnote{This is a sidenote that was entered
using the \texttt{\textbackslash footnote} command.}  If you'd like to place ancillary
information in the margin without the sidenote mark (the superscript
number), you can use the \doccmd{marginnote} command.\marginnote{This is a
margin note.  Notice that there isn't a number preceding the note, and
there is no number in the main text where this note was written.}

The specification of the \doccmddef{sidenote} command is:
\begin{docspec}
  \doccmd{sidenote}[\docopt{number}][\docopt{offset}]\{\docarg{Sidenote text.}\}
\end{docspec}

Both the \docopt{number} and \docopt{offset} arguments are optional.  If you
provide a \docopt{number} argument, then that number will be used as the
sidenote number.  It will change the number of the current sidenote only and
will not affect the numbering sequence of subsequent sidenotes.

Sometimes a sidenote may run over the top of other text or graphics in the
margin space.  If this happens, you can adjust the vertical position of the
sidenote by providing a dimension in the \docopt{offset} argument.  Some
examples of valid dimensions are:
\begin{docspec}
  \ttfamily 1.0in \qquad 2.54cm \qquad 254mm \qquad 6\Verb|\baselineskip|
\end{docspec}
If the dimension is positive it will push the sidenote down the page; if the
dimension is negative, it will move the sidenote up the page.

While both the \docopt{number} and \docopt{offset} arguments are optional, they
must be provided in order.  To adjust the vertical position of the sidenote
while leaving the sidenote number alone, use the following syntax:
\begin{docspec}
  \doccmd{sidenote}[][\docopt{offset}]\{\docarg{Sidenote text.}\}
\end{docspec}
The empty brackets tell the \Verb|\sidenote| command to use the default
sidenote number.

If you \emph{only} want to change the sidenote number, however, you may
completely omit the \docopt{offset} argument:
\begin{docspec}
  \doccmd{sidenote}[\docopt{number}]\{\docarg{Sidenote text.}\}
\end{docspec}

The \doccmddef{marginnote} command has a similar \docarg{offset} argument:
\begin{docspec}
  \doccmd{marginnote}[\docopt{offset}]\{\docarg{Margin note text.}\}
\end{docspec}

\section{References}
References are placed alongside their citations as sidenotes,
as well.  This can be accomplished using the normal \doccmddef{cite}
command.\sidenote{The first paragraph of this document includes a citation.}

The complete list of references may also be printed automatically by using
the \doccmddef{bibliography} command.  (See the end of this document for an
example.)  If you do not want to print a bibliography at the end of your
document, use the \doccmddef{nobibliography} command in its place.  

To enter multiple citations at one location,\cite{Tufte2006,Tufte1990} you can
provide a list of keys separated by commas and the same optional vertical
offset argument: \Verb|\cite{Tufte2006,Tufte1990}|.  
\begin{docspec}
  \doccmd{cite}[\docopt{offset}]\{\docarg{bibkey1,bibkey2,\ldots}\}
\end{docspec}

\section{Figures and Tables}\label{sec:figures-and-tables}
Images and graphics play an integral role in Tufte's work.
In addition to the standard \docenvdef{figure} and \docenvdef{tabular} environments,
this style provides special figure and table environments for full-width
floats.

Full page--width figures and tables may be placed in \docenvdef{figure*} or
\docenvdef{table*} environments.  To place figures or tables in the margin,
use the \docenvdef{marginfigure} or \docenvdef{margintable} environments as follows
(see figure~\ref{fig:marginfig}):

\begin{marginfigure}%
  \includegraphics[width=\linewidth]{helix}
  \caption{This is a margin figure.  The helix is defined by 
    $x = \cos(2\pi z)$, $y = \sin(2\pi z)$, and $z = [0, 2.7]$.  The figure was
    drawn using Asymptote (\url{http://asymptote.sf.net/}).}
  \label{fig:marginfig}
\end{marginfigure}

\begin{docspec}
\textbackslash begin\{marginfigure\}\\
  \qquad\textbackslash includegraphics\{helix\}\\
  \qquad\textbackslash caption\{This is a margin figure.\}\\
  \qquad\textbackslash label\{fig:marginfig\}\\
\textbackslash end\{marginfigure\}\\
\end{docspec}

The \docenv{marginfigure} and \docenv{margintable} environments accept an optional parameter \docopt{offset} that adjusts the vertical position of the figure or table.  See the ``\nameref{sec:sidenotes}'' section above for examples.  The specifications are:
\begin{docspec}
  \textbackslash{begin\{marginfigure\}[\docopt{offset}]}\\
  \qquad\ldots\\
  \textbackslash{end\{marginfigure\}}\\
  \mbox{}\\
  \textbackslash{begin\{margintable\}[\docopt{offset}]}\\
  \qquad\ldots\\
  \textbackslash{end\{margintable\}}\\
\end{docspec}

Figure~\ref{fig:fullfig} is an example of the \docenv{figure*}
environment and figure~\ref{fig:textfig} is an example of the normal
\docenv{figure} environment.

\begin{figure*}[h]
  \includegraphics[width=\linewidth]{sine.pdf}%
  \caption{This graph shows $y = \sin x$ from about $x = [-10, 10]$.
  \emph{Notice that this figure takes up the full page width.}}%
  \label{fig:fullfig}%
\end{figure*}

\begin{figure}
  \includegraphics{hilbertcurves.pdf}
%  \checkparity This is an \pageparity\ page.%
  \caption[Hilbert curves of various degrees $n$.][6pt]{Hilbert curves of various degrees $n$. \emph{Notice that this figure only takes up the main textblock width.}}
  \label{fig:textfig}
  %\zsavepos{pos:textfig}
\end{figure}

As with sidenotes and marginnotes, a caption may sometimes require vertical
adjustment. The \doccmddef{caption} command now takes a second optional
argument that enables you to do this by providing a dimension \docopt{offset}.
You may specify the caption in any one of the following forms:
\begin{docspec}
  \doccmd{caption}\{\docarg{long caption}\}\\
  \doccmd{caption}[\docarg{short caption}]\{\docarg{long caption}\}\\
  \doccmd{caption}[][\docopt{offset}]\{\docarg{long caption}\}\\
  \doccmd{caption}[\docarg{short caption}][\docopt{offset}]%
                  \{\docarg{long caption}\}
\end{docspec}
A positive \docopt{offset} will push the caption down the page. The short
caption, if provided, is what appears in the list of figures/tables, otherwise
the ``long'' caption appears there. Note that although the arguments
\docopt{short caption} and \docopt{offset} are both optional, they must be
provided in order. Thus, to specify an \docopt{offset} without specifying a
\docopt{short caption}, you must include the first set of empty brackets
\Verb|[]|, which tell \doccmd{caption} to use the default ``long'' caption. As
an example, the caption to figure~\ref{fig:textfig} above was given in the form
\begin{docspec}
  \doccmd{caption}[Hilbert curves...][6pt]\{Hilbert curves...\}
\end{docspec}

Table~\ref{tab:normaltab} shows table created with the \docpkg{booktabs}
package.  Notice the lack of vertical rules---they serve only to clutter
the table's data.


\newthought{Occasionally} \LaTeX{} will generate an error message:\label{err:too-many-floats}
\begin{docspec}
  Error: Too many unprocessed floats
\end{docspec}
\LaTeX{} tries to place floats in the best position on the page.  Until it's
finished composing the page, however, it won't know where those positions are.
If you have a lot of floats on a page (including sidenotes, margin notes,
figures, tables, etc.), \LaTeX{} may run out of ``slots'' to keep track of them
and will generate the above error.

\LaTeX{} initially allocates 18 slots for storing floats.  To work around this
limitation, the document classes provide a \doccmddef{morefloats} command
that will reserve more slots.

The first time \doccmd{morefloats} is called, it allocates an additional 34
slots.  The second time \doccmd{morefloats} is called, it allocates another 26
slots.

The \doccmd{morefloats} command may only be used two times.  Calling it a
third time will generate an error message.  (This is because we can't safely
allocate many more floats or \LaTeX{} will run out of memory.)

If, after using the \doccmd{morefloats} command twice, you continue to get the
\texttt{Too many unprocessed floats} error, there are a couple things you can
do.

The \doccmddef{FloatBarrier} command will immediately process all the floats
before typesetting more material.  Since \doccmd{FloatBarrier} will start a new
paragraph, you should place this command at the beginning or end of a
paragraph.

The \doccmddef{clearpage} command will also process the floats before
continuing, but instead of starting a new paragraph, it will start a new page.

You can also try moving your floats around a bit: move a figure or table to the
next page or reduce the number of sidenotes.  (Each sidenote actually uses
\emph{two} slots.)

After the floats have placed, \LaTeX{} will mark those slots as unused so they
are available for the next page to be composed.

\section{Captions}
You may notice that the captions are sometimes misaligned.
Due to the way \LaTeX's float mechanism works, we can't know for sure where it
decided to put a float. Therefore, the document classes provide commands to
override the caption position.

\paragraph{Vertical alignment} To override the vertical alignment, use the
\doccmd{setfloatalignment} command inside the float environment.  For
example:

\begin{fullwidth}
\begin{docspec}
  \textbackslash begin\{figure\}[btp]\\
  \qquad \textbackslash includegraphics\{sinewave\}\\
  \qquad \textbackslash caption\{This is an example of a sine wave.\}\\
  \qquad \textbackslash label\{fig:sinewave\}\\
  \qquad \hlred{\textbackslash setfloatalignment\{b\}\% forces caption to be bottom-aligned}\\
  \textbackslash end\{figure\}
\end{docspec}
\end{fullwidth}

\noindent The syntax of the \doccmddef{setfloatalignment} command is:

\begin{docspec}
  \doccmd{setfloatalignment}\{\docopt{pos}\}
\end{docspec}

\noindent where \docopt{pos} can be either \texttt{b} for bottom-aligned
captions, or \texttt{t} for top-aligned captions.

\paragraph{Horizontal alignment}\label{par:overriding-horizontal}
To override the horizontal alignment, use either the \doccmd{forceversofloat}
or the \doccmd{forcerectofloat} command inside of the float environment.  For
example:

\begin{fullwidth}
\begin{docspec}
  \textbackslash begin\{figure\}[btp]\\
  \qquad \textbackslash includegraphics\{sinewave\}\\
  \qquad \textbackslash caption\{This is an example of a sine wave.\}\\
  \qquad \textbackslash label\{fig:sinewave\}\\
  \qquad \hlred{\textbackslash forceversofloat\% forces caption to be set to the left of the float}\\
  \textbackslash end\{figure\}
\end{docspec}
\end{fullwidth}

The \doccmddef{forceversofloat} command causes the algorithm to assume the
float has been placed on a verso page---that is, a page on the left side of a
two-page spread.  Conversely, the \doccmddef{forcerectofloat} command causes
the algorithm to assume the float has been placed on a recto page---that is, a
page on the right side of a two-page spread.


\section{Full-width text blocks}

In addition to the new float types, there is a \docenvdef{fullwidth}
environment that stretches across the main text block and the sidenotes
area.

\begin{Verbatim}
\begin{fullwidth}
Lorem ipsum dolor sit amet...
\end{fullwidth}
\end{Verbatim}

\begin{fullwidth}
The news and popular culture want us to fear the robot apocolypse.  Newspapers warn us that robots are coming for our jobs, and science fiction claims that the robots are coming for our lives.  At the same time, researchers building artificial intelligence are heralding their sucesses at Go,\cite{silver-16} Starcraft,\cite{vinyals2017starcraft} and Jeopardy!:\cite{ferruci-10} a revolution is around the corner.
\end{fullwidth}

\section{Typography}\label{sec:typography}

\subsection{Typefaces}\label{sec:typefaces}\index{typefaces}
If the Palatino, \textsf{Helvetica}, and \texttt{Bera Mono} typefaces are installed, this style
will use them automatically.  Otherwise, we'll fall back on the Computer Modern
typefaces.

\subsection{Letterspacing}\label{sec:letterspacing}
This document class includes two new commands and some improvements on
existing commands for letterspacing.

When setting strings of \allcaps{ALL CAPS} or \smallcaps{small caps}, the
letter\-spacing---that is, the spacing between the letters---should be
increased slightly.\cite{Bringhurst2005}  The \doccmddef{allcaps} command has proper letterspacing for
strings of \allcaps{FULL CAPITAL LETTERS}, and the \doccmddef{smallcaps} command
has letterspacing for \smallcaps{small capital letters}.  These commands
will also automatically convert the case of the text to upper- or
lowercase, respectively.

The \doccmddef{textsc} command has also been redefined to include
letterspacing.  The case of the \doccmd{textsc} argument is left as is,
however.  This allows one to use both uppercase and lowercase letters:
\textsc{The Initial Letters Of The Words In This Sentence Are Capitalized.}



\section{Document Class Options}\label{sec:options}

\index{class options|(}
The \doccls{tufte-book} class is based on the \LaTeX\ \doccls{book}
document class.  Therefore, you can pass any of the typical book
options.  There are a few options that are specific to the
\doccls{tufte-book} document class, however.

The \docclsoptdef{a4paper} option will set the paper size to \smallcaps{A4} instead of
the default \smallcaps{US} letter size.

The \docclsoptdef{sfsidenotes} option will set the sidenotes and title block in a 
\textsf{sans serif} typeface instead of the default roman.

The \docclsoptdef{twoside} option will modify the running heads so that the page
number is printed on the outside edge (as opposed to always printing the page
number on the right-side edge in \docclsoptdef{oneside} mode).  

The \docclsoptdef{symmetric} option typesets the sidenotes on the outside edge of
the page.  This is how books are traditionally printed, but is contrary to
Tufte's book design which sets the sidenotes on the right side of the page.
This option implicitly sets the \docclsopt{twoside} option.

The \docclsoptdef{justified} option sets all the text fully justified (flush left
and right).  The default is to set the text ragged right.  
The body text of Tufte's books are set ragged right.  This prevents
needless hyphenation and makes it easier to read the text in the slightly
narrower column.

The \docclsoptdef{bidi} option loads the \docpkg{bidi} package which is used with
\tXeLaTeX\ to typeset bi-directional text.  Since the \docpkg{bidi}
package needs to be loaded before the sidenotes and cite commands are defined,
it can't be loaded in the document preamble.

The \docclsoptdef{debug} option causes the classes to output debug
information to the log file which is useful in troubleshooting bugs.  It will
also cause the graphics to be replaced by outlines.

The \docclsoptdef{nofonts} option prevents the classes from
automatically loading the Palatino and Helvetica typefaces.  You should use
this option if you wish to load your own fonts.  If you're using \tXeLaTeX, this
option is implied (i.e., the Palatino and Helvetica fonts aren't loaded if you
use \tXeLaTeX).  

The \docclsoptdef{nols} option inhibits the letterspacing code.  The \TL\
classes try to load the appropriate letterspacing package (either pdf\TeX's
\docpkg{letterspace} package or the \docpkg{soul} package).  If you're using
\tXeLaTeX\ with \docpkg{fontenc}, however, you should configure your own
letterspacing.  

The \docclsoptdef{notitlepage} option causes \doccmd{maketitle} to generate a title
block instead of a title page.  The \doccls{book} class defaults to a title
page and the \doccls{handout} class defaults to the title block.  There is an
analogous \docclsoptdef{titlepage} option that forces \doccmd{maketitle} to
generate a full title page instead of the title block.

The \docclsoptdef{notoc} option suppresses \TL's custom table of contents
(\textsc{toc}) design.  The current \textsc{toc} design only shows unnumbered
chapter titles; it doesn't show sections or subsections.  The \docclsopt{notoc}
option will revert to \LaTeX's \textsc{toc} design.

The \docclsoptdef{nohyper} option prevents the \docpkg{hyperref} package from
being loaded.  The default is to load the \docpkg{hyperref} package and use the
\doccmd{title} and \doccmd{author} contents as metadata for the generated
\textsc{pdf}.

\index{class options|)}



\chapter[Customizing Tufte-LaTeX]{Customizing \TL}
\label{ch:customizing}

The document classes are designed to closely emulate Tufte's book
design by default.  However, each document is different and you may encounter
situations where the default settings are insufficient.  This chapter explores
many of the ways you can adjust the document classes to better fit
your needs.

\section{File Hooks}
\label{sec:filehooks}

\index{file hooks|(}
If you create many documents using the classes, it's easier to
store your customizations in a separate file instead of copying them into the
preamble of each document.  The classes provide three file hooks:
\docfilehook{tufte-common-local.tex}{common}, \docfilehook{tufte-book-local.tex}{book}, and
\docfilehook{tufte-handout-local.tex}{handout}.\sloppy

\begin{description}
  \item[\docfilehook{tufte-common-local.tex}{common}]
    If this file exists, it will be loaded by all of the document
    classes just prior to any document-class-specific code.  If your
    customizations or code should be included in both the book and handout
    classes, use this file hook.
  \item[\docfilehook{tufte-book-local.tex}{book}] 
    If this file exists, it will be loaded after all of the common and
    book-specific code has been read.  If your customizations apply only to the
    book class, use this file hook.
  \item[\docfilehook{tufte-common-handout.tex}{handout}] 
    If this file exists, it will be loaded after all of the common and
    handout-specific code has been read.  If your customizations apply only to
    the handout class, use this file hook.
\end{description}

\index{file hooks|)}

\section{Numbered Section Headings}
\label{sec:numbered-sections}
\index{headings!numbered}

While Tufte dispenses with numbered headings in his books, if you require them,
they can be anabled by changing the value of the \doccounter{secnumdepth}
counter.  From the table below, select the heading level at which numbering
should stop and set the \doccounter{secnumdepth} counter to that value.  For
example, if you want parts and chapters numbered, but don't want numbering for
sections or subsections, use the command:
\begin{docspec}
  \doccmd{setcounter}\{secnumdepth\}\{0\}
\end{docspec}

The default \doccounter{secnumdepth} for the document classes is $-1$.

\begin{table}
  \footnotesize
  \begin{center}
    \begin{tabular}{lr}
      \toprule
      Heading level & Value \\
      \midrule
      Part (in \doccls{tufte-book}) & $-1$ \\
      Part (in \doccls{tufte-handout}) & $0$ \\
      Chapter (only in \doccls{tufte-book}) & $0$ \\
      Section & $1$ \\
      Subsection & $2$ \\
      Subsubsection & $3$ \\
      Paragraph & $4$ \\
      Subparagraph & $5$ \\
      \bottomrule
    \end{tabular}
  \end{center}
  \caption{Heading levels used with the \texttt{secnumdepth} counter.}
\end{table}

\section{Changing the Paper Size}
\label{sec:paper-size}

The classes currently only provide three paper sizes: \textsc{a4},
\textsc{b5}, and \textsc{us} letter.  To specify a different paper size (and/or
margins), use the \doccmd[geometry]{geometry} command in the preamble of your
document (or one of the file hooks).  The full documentation of the
\doccmd{geometry} command may be found in the \docpkg{geometry} package
documentation.\cite{pkg-geometry}


\section{Customizing Marginal Material}
\label{sec:marginal-material}

Marginal material includes sidenotes, citations, margin notes, and captions.
Normally, the justification of the marginal material follows the justification
of the body text.  If you specify the \docclsopt{justified} document class
option, all of the margin material will be fully justified as well.  If you
don't specify the \docclsopt{justified} option, then the marginal material will
be set ragged right.

You can set the justification of the marginal material separately from the body
text using the following document class options: \docclsopt{sidenote},
\docclsopt{marginnote}, \docclsopt{caption}, \docclsopt{citation}, and
\docclsopt{marginals}.  Each option refers to its obviously corresponding
marginal material type.  The \docclsopt{marginals} option simultaneously sets
the justification on all four marginal material types.

Each of the document class options takes one of five justification types:
\begin{description}
  \item[\docclsopt{justified}] Fully justifies the text (sets it flush left and
    right).
  \item[\docclsopt{raggedleft}] Sets the text ragged left, regardless of which
    page it falls on.
  \item[\docclsopt{raggedright}] Sets the text ragged right, regardless of
    which page it falls on.
  \item[\doccls{raggedouter}] Sets the text ragged left if it falls on the
    left-hand (verso) page of the spread and otherwise sets it ragged right.
    This is useful in conjunction with the \docclsopt{symmetric} document class
    option.
  \item[\docclsopt{auto}] If the \docclsopt{justified} document class option
    was specified, then set the text fully justified; otherwise the text is set
    ragged right.  This is the default justification option if one is not
    explicitly specified.
\end{description}

\noindent For example, 
\begin{docspec}
  \doccmdnoindex{documentclass}[symmetric,justified,marginals=raggedouter]\{tufte-book\}
\end{docspec}
will set the body text of the document to be fully justified and all of the
margin material (sidenotes, margin notes, captions, and citations) to be flush
against the body text with ragged outer edges.

\newthought{The font and style} of the marginal material may also be modified using the following commands:

\begin{docspec}
  \doccmd{setsidenotefont}\{\docopt{font commands}\}\\
  \doccmd{setcaptionfont}\{\docopt{font commands}\}\\
  \doccmd{setmarginnotefont}\{\docopt{font commands}\}\\
  \doccmd{setcitationfont}\{\docopt{font commands}\}
\end{docspec}

The \doccmddef{setsidenotefont} sets the font and style for sidenotes, the
\doccmddef{setcaptionfont} for captions, the \doccmddef{setmarginnotefont} for
margin notes, and the \doccmddef{setcitationfont} for citations.  The
\docopt{font commands} can contain font size changes (e.g.,
\doccmdnoindex{footnotesize}, \doccmdnoindex{Huge}, etc.), font style changes (e.g.,
\doccmdnoindex{sffamily}, \doccmdnoindex{ttfamily}, \doccmdnoindex{itshape}, etc.), color changes (e.g.,
\doccmdnoindex{color}\texttt{\{blue\}}), and many other adjustments.

If, for example, you wanted the captions to be set in italic sans serif, you could use:
\begin{docspec}
  \doccmd{setcaptionfont}\{\doccmdnoindex{itshape}\doccmdnoindex{sffamily}\}
\end{docspec}


%%
% The back matter contains appendices, bibliographies, indices, glossaries, etc.





\newcommand{\outline}[3]{\chapter{#1}
  \begin{itemize}
  \item {\bf Story:} #2
  \item {\bf Essay:} #3
  \end{itemize}
}

\outline{General AI}{\lastname{} is called in to consult on the
  deployment of a general AI that a military contractor has developed.
  It breaks out of its sandbox, causing physical havoc with military
  equipment.  A specialized AI designed to contain the general AI is
  surgically deployed to resolve the problem.}{This chapter explores
  the distinction between general and specialized
  AI---\specializedai{}---and how the pressures and history of a
  general AI cause them to be vulnerable to specialized algorithms and
  methods.

  With respect to current day, the chapter reviews the approaches
  toward general AI (and how they haven't gotten anywhere) as opposed
  to specialized AI.  }

\outline{Resource-Constrained AI}{\lastname{} is comfortably faculty
  at University when her past history with \specializedai{} comes back
  to haunt her: a rogue actor has found an old copy and using it to
  terrorize a small Canadian town.  \lastname{} works with the local
  authorities to deploy their specialized AI to counteract the
  threat.}{This chapter explores how governments' role as a protector
  of populations evolves in a world with ubiquitous and powerful
  artifcial intelligence.  Although in some ways AI is democratizing,
  it is still connected to access to real-world resources (material,
  energy, physical space), and governments and multinational
  corporations will have the most powerful and capable AI agents.
  Just as we trust governments with technology that can end the world
  (nuclear weapons), we must trust the government to responsibly
  harness the power of artificial intelligence.

  This discussion will be connected to the connection to electricity
  and specialized hardware for bitcoining mining and deep learning.  }

\outline{AI's Role in Social Interactions}{\lastname{} finds herself
  drawn into a political scandal when an AI agent begins influencing
  public opinion against her and her research.  She fights back the
  only way she knows how: with AI.}{While the threat of AI is often
  seen as physical, another salient fear is social and emotional: AIs
  can play on fear and emotion.  In many ways this is harder to notice
  and fight against because these effects might be superficially
  welcome by their targets.  This chapter talks about how AIs can
  detect, influence, and interact with people.

  Examples will be drawn from AI approaches to detecting deception and
  manipulating audiences on social media.  }

\jbgcomment{Not putting in outline, but what might be a way of making
  the plot work, this could be a scheme by a faculty adversary that
  she uncovers, which leads her to sell out.}

\outline{Cashing Out}{Fed up by politics and the politics of academia,
  \lastname{} goes off to make a pile of money at \fintech{}, turning
  her AI smarts to Wall Street.  It's harder than she
  thinks.}{Financial markets, like much of AI, hope to make
  predictions about what will happen in the world and to estimate
  uncertainty.  This chapter investigates whether capitalist markets
  are possible in a world with effective AI.
}

\jbgcomment{Connection between the two chapters: previous chapter allows everyone to go out an earn a capitalist wage.}

\outline{The Silent Revolution}{Having saved the world multiple times,
  \lastname{} retires to spend more time with her grandchildren and
  sees their relationship with AI.}{The book concludes with how AI can
  change social interactions and society in less overtly negative
  ways, warning about a future where AI can diminish the richness of
  the human experience (if we let it).

  More of an opinion piece, this chapter focuses on how AI and humans
  should interact with each other.  Specifically, arguing for
  cooperation and collaboration rather than outright replacement.  }

\backmatter

\bibliography{bib/jbg}
\bibliographystyle{plainnat}


\printindex

\end{document}
